\chapter{Практическая часть}

\textbf{Задание 17:} Используя хвостовую рекурсию, разработать, комментируя аргументы, эффективную программу, позволяющую:
\begin{itemize}
    \item сформировать список из элементов числового списка, больших заданного
значения;
    \item сформировать список из элементов, стоящих на нечетных позициях исходного
списка (нумерация от 0);
    \item удалить заданный элемент из списка (один или все вхождения);
    \item преобразовать список в множество (можно использовать ранее разработанные процедуры).
\end{itemize}

Убедиться в правильности результатов

\textbf{Для одного} из вариантов \textbf{ВОПРОСА} и одного из заданий составить таблицу, отражающую конкретный порядок работы системы.

Т.к. резольвента хранится в виде стека, то состояние резольвенты требуется отображать в столбик: вершина – сверху! Новый шаг надо начинать с нового состояния резольвенты! Для каждого запуска алгоритма унификации, требуется указать № выбранного правила и дальнейшие действия – и почему.

\clearpage

\listingfile{1.pro}{1-1}{Prolog}{Реализация программы для задания 1}{linerange={1-18}}

\listingfile{2.pro}{1-2}{Prolog}{Реализация программы для задания 2}{linerange={1-13}}

\clearpage

\listingfile{3.pro}{1-2}{Prolog}{Реализация программы для задания 3}{linerange={1-25}}

В Таблице \ref{tbl:1-1} представлен порядок поиска ответа на вопрос 1.

\begin{landscape}
    \setlength{\LTcapwidth}{\linewidth}
    \begin{longtable}{|c|c|c|c|c|}
        \caption[Порядок формирования результата для 1-го вопроса]{Порядок формирования результата для 1-го вопроса} \label{tbl:1-1}\\
    
        \hline
            Шаг & Сравниваемые термы; & Дальнейшие & Резольвента & Подстановка \\
                & результаты & действия & & \\
        \endfirsthead
    
        \multicolumn{5}{l}
        {{\tablename\ \thetable{} -- продолжение}} \\
        \hline 
            Шаг & Сравниваемые термы; & Дальнейшие & Резольвента & Подстановка \\
                & результаты & действия & & \\\hline
        \endhead
        
        \hline \multicolumn{5}{|r|}{{Продолжение на следующей странице}} \\ \hline
        \endfoot
        
        \hline \multicolumn{5}{|r|}{{Конец таблицы}} \\ \hline
        \endlastfoot
        
        \hline
              & f([3, 6, 0, -1, 4], 3, R). & Прямой ход & 3 > 3 & H = 3\\
            1 & и f([H|T], El, [H|Res]) & & ! & T = [6, 0, -1, 4]\\
			  & & & f([6, 0, -1, 4], 3, Res) & El = 3\\
			\hline
              & 3 > 3 & Откат & ! & H = 3\\
            2 & & & f([6, 0, -1, 4], 3, Res) & T = [6, 0, -1, 4]\\
			  & & & & El = 3\\
			\hline
              & f([3, 6, 0, -1, 4], 3, R). & Прямой ход & f([6, 0, -1, 4], 3, R) & T = [6, 0, -1, 4]\\
            3 & и f([\_|T], El, [H|Res]) & & & El = 3\\
			  & & & & \\
			\hline
			  & f([6, 0, -1, 4], 3, R). & Прямой ход & 6 > 3 & H = 6\\
            4 & и f([H|T], El, [H|Res]) & & ! & T = [0, -1, 4]\\
			  & & & f([0, -1, 4], 3, Res) & El = 3\\
			\hline
			  & 6 > 3 Ю 3 & Прямой ход & ! & H = 6\\
            5 & & & f([0, -1, 4], 3, Res) & T = [0, -1, 4]\\
			  & & &  & El = 3\\
			\hline
              & ! Ю 3 & Прямой ход & f([0, -1, 4], 3, Res) & H = 6\\
            6 & & & & T = [0, -1, 4]\\
			  & & & & El = 3\\
			\hline
			\hline
			\dots & \dots & \dots & \dots & \dots \\
			\hline 
			18 & f([], 3, []) & Прямой ход & ! & Res = [6, 4]\\
              & и f([], \_, []) & & &\\
            \hline 
			19 & ! & Завершение & ! & Res = [6, 4]\\
              & & 1 подст. & &\\
              & & в рез-те & &\\
    \end{longtable}
\end{landscape}