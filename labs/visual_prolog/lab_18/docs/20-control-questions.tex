\chapter{Контрольный вопросы}

\section{Что такое рекурсия?}

Рекурсия – это ссылка на описываемый объект при описании объекта.

\section{Как организуется хвостовая рекурсия в Prolog?}

\begin{itemize}
    \item рекурсивный вызов единственен и расположен в конце тела правила;
    \item не должно быть возможности сделать откат до вычисления рекурсивного вызова.
\end{itemize}

\section{Как организовать выход из рекурсии в Prolog?}

С помощью отсечения

\section{Какое первое состояние резольвенты?}

Заданный вопрос (goal).

\section{В каких пределах программы переменные уникальны?}

Именованная переменная уникальна в рамках предложения, в котором она используется. Анонимные переменные всегда уникальны.

\section{В какой момент, и каким способом системе удается получить доступ к голове списка?}

Получить голову или хвост списка можно при унификации списка с [H|T], H -- голова списка, T -– хвост списка.

\section{Каково назначение и результат использования алгоритма унификации?}

Унификация -– механизм логического вывода. Результат –- подстановка.

\section{Как формируется новое состояние резольвенты?}

Преобразования резольвенты выполняются с помощью редукции. Редукцией цели G с помощью программы P называется замена цели G телом того правила из P, заголовок которого унифицируется с целью. Новая резольвента образуется в два этапа:
\begin{itemize}
    \item в текущей резольвенте выбирается одна из подцелей и для неё выполняется редукция;
    \item к полученной конъюнкции целей применяется подстановка, полученная как наибольший общий унификатор цели и заголовка сопоставленного с ней правила.
\end{itemize}

\section{Как применяется подстановка, полученная с помощью алгоритма унификации? Как глубоко?}

Подстановка применяется к целям в резольвенте путем замены текущей переменной на соответствующий терм. В результате применения подстановки некоторые переменные конкретизируются значениями, которые (значения) могут и будут далее использованы при доказательстве истинности тела выбранного правила.

\section{В каких случаях запускается механизм отката?}

Механизм отката запустится в случае неудачи алгоритма унификации.

\section{Когда останавливается работа системы?}

Работа системы останавливается, когда найдены все возможные ответы на вопрос.

\section{Как это определяется на формальном уровне?}

Когда в резольвенте находится исходный вопрос, для которого пройдена вся БЗ.