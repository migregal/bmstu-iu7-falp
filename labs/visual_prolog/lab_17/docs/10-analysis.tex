\chapter{Практическая часть}

\textbf{Задание 17:} Используя хвостовую рекурсию, разработать эффективную программу, (комментируя назначение аргументов), позволяющую:
\begin{itemize}
    \item найти длину списка (по верхнему уровню);
    \item найти сумму элементов числового списка;
    \item найти сумму элементов числового списка, стоящих на нечетных позициях исходного списка (нумерация от 0).
\end{itemize}

Убедиться в правильности результатов.

\textbf{Для одного} из вариантов \textbf{ВОПРОСА} и одного из заданий составить таблицу, отражающую конкретный порядок работы системы.

Т.к. резольвента хранится в виде стека, то состояние резольвенты требуется отображать в столбик: вершина – сверху! Новый шаг надо начинать с нового состояния резольвенты! Для каждого запуска алгоритма унификации, требуется указать № выбранного правила и дальнейшие действия – и почему.

\clearpage

\listingfile{lab17.pro}{1-1}{Prolog}{Реализация вычисления числа Фибоначчи}{linerange={1-36}}

В Таблице \ref{tbl:1-1} представлен порядок поиска ответа на вопрос 1.

\begin{landscape}
    \setlength{\LTcapwidth}{\linewidth}
    \begin{longtable}{|c|c|c|c|c|}
        \caption[Порядок формирования результата для 1-го вопроса]{Порядок формирования результата для 1-го вопроса} \label{tbl:1-1}\\
    
        \hline
            Шаг & Сравниваемые термы; & Дальнейшие & Резольвента & Подстановка \\
                & результаты & действия & & \\
        \endfirsthead
    
        \multicolumn{5}{l}
        {{\tablename\ \thetable{} -- продолжение}} \\
        \hline 
            Шаг & Сравниваемые термы; & Дальнейшие & Резольвента & Подстановка \\
                & результаты & действия & & \\\hline
        \endhead
        
        \hline \multicolumn{5}{|r|}{{Продолжение на следующей странице}} \\ \hline
        \endfoot
        
        \hline \multicolumn{5}{|r|}{{Конец таблицы}} \\ \hline
        \endlastfoot
        
        \hline
            1 & len([0, 1, -2, 10], Len) & Прямой ход & len([0, 1, -2, 10], 0, Len) & List = [0, 1, -2, 10]\\
             & и len(List, Len) & & &\\
			\hline
			 & len([0, 1, -2, 10], 0, Len) & Прямой ход & len([0, 1, -2, 10], 0, Len) & List = [0, 1, -2, 10]\\
            2 & и len(List, Len) & Переход к & &\\
             & Не унифицируемы & след. предл. & &\\
            \hline
			&&&&\\
			\dots & \dots & \dots & \dots & \dots \\
			&&&&\\
			\hline 
			4 & len([0, 1, -2, 10], 0, Len) & Прямой ход & NewLen = CurLen + 1 & T = [1, -2, 10]\\
              & и len([\_|T], CurLen, Len) & & len([1, -2, 10], NewLen, Len) & CurLen = 0\\
            \hline 
		     & NewLen = CurLen + 1 & Прямой ход & len([1, -2, 10], 1, Len) & T = [1, -2, 10]\\
            5 & & & & CurLen = 0\\
             & & & & NewLen = 1\\
            \hline
			&&&&\\
			\dots & \dots & \dots & \dots & \dots \\
			&&&&\\
            \hline 
		    8 & len([1, -2, 10], 1, Len) & Прямой ход & NewLen = CurLen + 1 & T = [-2, 10]\\
              & и len([\_|T], CurLen, Len) & & len([-2, 10], NewLen, Len) & CurLen = 1\\
            \hline 
		    9 & NewLen = CurLen + 1 & Прямой ход & len([-2, 10], NewLen, Len) & T = [-2, 10]\\
              & & &  & NewLen = 2\\
            \hline
			&&&&\\
			\dots & \dots & \dots & \dots & \dots \\
			&&&&\\
			\hline 
		    19 & len([], 4, Len) & прямой ход & ! & T = []\\
              & и len([], Len, Len) & &  & NewLen = 4\\
              & & &  & Len = 4\\
            \hline 
		    20 & ! & Завершение работы & & Len = 4\\
              & & 1 подст. &  & \\
              & & в рез-те & & \\
    \end{longtable}
\end{landscape}