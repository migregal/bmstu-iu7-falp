\chapter{Контрольный вопросы}

\section{В каком фрагменте программы сформулировано знание? Это знание о чем на формальном уровне?}

Знания сформулированы в \texttt{clauses} (факты и правила). Это знания о предметной
области.

\section{Что содержит тело правила?}

В заголовке правила находится знание о предметной области, а в теле
содержится условия истинности этого знания.

\section{Что дает использование переменных при формулировании знаний? В чем отличие формулировки знания с помощью термов с одинаковой арностью при использовании одной переменной и при использовании нескольких
переменных?}

Связанная с каким-то значением переменная, в рамках одного предложения,
может быть использована в других местах.
Чем больше переменных содержит формулировка правила, тем более общим
будет являться терм.

\section{С каким квантором переменные входят в правило, в каких пределах переменная уникальна?}

Переменные входят в правило с квантором всеобщности. Именованная
переменная уникальна в рамках предложения, в котором она используется.

\section{Какова семантика (смысл) предложений раздела DOMAINS? Когда, где и с какой целью используется это описание?}

DOMAINS -- раздел описания доменов. Этот раздел используется для описания
используемых структур данных.

\section{Какова семантика (смысл) предложений раздела PREDICATES? Когда, и где используется это описание?}
PREDICATES -- раздел описания предикатов. Это описание используется для
проверки корректности «типов» знаний.

\section{Унификация каких термов запускается на самом первом шаге работы
системы?}

Вопроса и первого терма в \texttt{clauses}.

\section{Каковы назначение и результат использования алгоритма унификации?}

Унификация –- попытка сопоставить два терма. Результат: успех/неудача.

\section{В каком случае запускается механизм отката?}

Механизм отката запустится в случае неудачи алгоритма унификации.