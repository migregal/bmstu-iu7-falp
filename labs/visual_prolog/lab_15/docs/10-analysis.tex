\chapter{Практическая часть}

\textbf{Задание 15:} В одной программе написать правила, позволяющие найти
\begin{enumerate}
    \item Максимум из двух чисел
    \begin{enumerate}
        \item без использования отсечения,
        \item с использованием отсечения;
    \end{enumerate}
    \item Максимум из трех чисел
    \begin{enumerate}
        \item без использования отсечения,
        \item с использованием отсечения;
    \end{enumerate}
\end{enumerate}

Убедиться в правильности результатов.
Для каждого случая пункта 2 обосновать необходимость всех условий тела.
Для одного из вариантов ВОПРОСА и каждого варианта задания 2 составить
таблицу, отражающую конкретный порядок работы системы:
Т.к. резольвента хранится в виде стека, то состояние резольвенты требуется
отображать в столбик: вершина – сверху! Новый шаг надо начинать с нового
состояния резольвенты!

\clearpage

\listingfile{main.pro}{1-1}{Prolog}{Реализация базы знаний}{linerange={1-26}}

В Таблицах \ref{tbl:1-1}-\ref{tbl:1-2} представлен порядок поиска ответа на вопросы 1 и 2, соответственно.

\begin{landscape}
    \setlength{\LTcapwidth}{\linewidth}
    \begin{longtable}{|c|c|c|c|c|}
        \caption[Порядок формирования результата для 1-го вопроса]{Порядок формирования результата для 1-го вопроса} \label{tbl:1-1}\\
    
        \hline
            Шаг & Сравниваемые термы; & Дальнейшие & Резольвента & Подстановка \\
                & результаты & действия & & \\
        \endfirsthead
    
        \multicolumn{5}{l}
        {{\tablename\ \thetable{} -- продолжение}} \\
        \hline 
            Шаг & Сравниваемые термы; & Дальнейшие & Резольвента & Подстановка \\
                & результаты & действия & & \\\hline
        \endhead
        
        \hline \multicolumn{5}{|r|}{{Продолжение на следующей странице}} \\ \hline
        \endfoot
        
        \hline \multicolumn{5}{|r|}{{Конец таблицы}} \\ \hline
        \endlastfoot
        
        \hline
              & maxThree(3, 1, 2, R) & Прямой ход & maxThree(3, 1, 2, R) & \\
            1 & и maxTwo(A, B, A) & Переход к & &\\
			  & Главные функторы не равны & след. предл. & &\\
			\hline
			\dots & \dots & \dots & \dots & \dots \\
			\hline 
			5 & maxThree(3, 1, 2, R) и maxThree(A, B, C, A) & Прямой ход & 3 >= 1, 3 >= 2 & A = 3, B = 1, C = 2, R = 3\\
            \hline 
			6 & 3 >= 1 & Прямой ход & 3 >= 2  & A = 3, B = 1, C = 2, R = 3\\
            \hline 
			7 & 3 >= 2 & Нашли ответ & & A = 3, B = 1, C = 2, R = 3\\
            \hline 
			8 & maxThree(3, 1, 2, R) & Прямой ход & maxTwo(2, 1, Res) & B = 1\\
              & и maxThree(\_, B, C, Res) & & & C = 2 \\
            \hline 
			9 & maxTwo(2, 1, Res) & Прямой ход & A >= B & A = 2, B = 1\\
              & и maxTwo(A, B, A) & & & R = 2 \\
            \hline 
			10 & A >= B & Нашли ответ & & A = 2, B = 1, R = 2\\
			\hline
			\dots & \dots & \dots & \dots & \dots \\
			\hline
            30  & maxThree(3, 1, 2, R) & Завершение & maxThree(3, 1, 2, R) & \\
              & и maxThreeCut(\_, B, C, Res)& работы & &\\
              & Line, Sex) & 2 подст. & & \\
              & Главные функторы не равны & в рез-те & & \\
    \end{longtable}
\end{landscape}

\begin{landscape}
    \setlength{\LTcapwidth}{\linewidth}
    \begin{longtable}{|c|c|c|c|c|}
        \caption[Порядок формирования результата для 2-го вопроса]{Порядок формирования результата для 2-го вопроса} \label{tbl:1-2}\\
    
        \hline
            Шаг & Сравниваемые термы; & Дальнейшие & Резольвента & Подстановка \\
                & результаты & действия & & \\
        \endfirsthead
    
        \multicolumn{5}{l}
        {{\tablename\ \thetable{} -- продолжение}} \\
        \hline 
            Шаг & Сравниваемые термы; & Дальнейшие & Резольвента & Подстановка \\
                & результаты & действия & & \\\hline
        \endhead
        
        \hline \multicolumn{5}{|r|}{{Продолжение на следующей странице}} \\ \hline
        \endfoot
        
        \hline \multicolumn{5}{|r|}{{Конец таблицы}} \\ \hline
        \endlastfoot
        
        \hline
              & maxThreeCut(3, 1, 2, R) & Прямой ход & maxThreeCut(3, 1, 2, R) & \\
            1 & и maxTwo(A, B, A) & Переход к & &\\
			  & Главные функторы не равны & след. предл. & &\\
			\hline
			\dots & \dots & \dots & \dots & \dots \\
			\hline 
			7 & maxThreeCut(3, 1, 2, R) & Прямой ход & 3 >= 1, 3 >= 2, ! & A = 3, B = 1\\
			& и maxThreeCut(A, B, C, A) & & & C = 2, R = 3\\
            \hline 
			8 & 3 >= 1 & Прямой ход & 3 >= 1, ! & A = 3, B = 1, C = 2, R = 3\\
            \hline 
			9 & 3 >= 2 & Нашли ответ & ! & A = 3, B = 1, C = 2, R = 3\\
			\hline 
			10 & ! & Завершение & & A = 3, B = 1, C = 2, R = 3\\
			   & & работы & &\\
               & & 1 подст. & & \\
               & & в рез-те & & \\
    \end{longtable}
\end{landscape}
