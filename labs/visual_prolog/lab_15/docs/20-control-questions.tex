\chapter{Контрольный вопросы}

\section{Какое первое состояние резольвенты?}

Заданный вопрос (goal).

\section{В каком случае система запускает алгоритм унификации?}

Система запускает алгоритм унификации автоматически при необходимости что-то доказать

\section{Каково назначение и результат использования алгоритма унификации?}

Унификация – механизм логического вывода. Результат – подстановка.

\section{В каких пределах программы переменные уникальны?}

Именованная переменная уникальна в рамках предложения, в котором она используется. Анонимные переменные всегда уникальны.

\section{Как применяется подстановка, полученная с помощью алгоритма унификации?}

Подстановка применяется к целям в резольвенте путем замены текущей переменной на соответствующий терм.

\section{Как изменяется резольвента?}

Преобразования резольвенты выполняются с помощью редукции. Редукцией цели G с помощью программы P называется замена цели G телом того правила из P, заголовок которого унифицируется с целью. Новая резольвента образуется в два этапа:
\begin{itemize}
    \item в текущей резольвенте выбирается одна из подцелей и для неё выполняется редукция;
    \item к полученной конъюнкции целей применяется подстановка, полученная как наибольший общий унификатор цели и заголовка сопоставленного с ней правила.
\end{itemize}

\section{В каких случаях запускается механизм отката?}

Механизм отката запустится в случае неудачи алгоритма унификации.