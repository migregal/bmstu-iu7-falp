\chapter{Практическая часть}

\textbf{Задание 14:} Создать базу знаний: «ПРЕДКИ», позволяющую наиболее эффективным способом (за меньшее количество шагов, что обеспечивается меньшим количеством предложений БЗ - правил), используя разные варианты (примеры) одного вопроса, определить (указать: какой вопрос для какого варианта):

\begin{enumerate}
    \item по имени субъекта определить всех его бабушек (предки 2-го колена);
    \item по имени субъекта определить всех его дедушек (предки 2-го колена);
    \item по имени субъекта определить всех его бабушек и дедушек (предки 2-го колена);
    \item по имени субъекта определить его бабушку по материнской линии (предки 2-го колена);
    \item по имени субъекта определить его бабушку и дедушку по материнской линии (предки 2-го колена).
\end{enumerate}

Минимизировать количество правил и количество вариантов вопросов. Использовать \textbf{конъюнктивные правила и простой вопрос}.

\textbf{Для одного} из вариантов \textbf{ВОПРОСА} и конкретной БЗ \textbf{составить таблицу},
отражающую конкретный порядок работы системы, с объяснениями:
\begin{itemize}
    \item очередная проблема на каждом шаге и метод ее решения;
    \item каково новое текущее состояние резольвенты, как получено;
    \item какие дальнейшие действия? (Запускается ли алгоритм унификации? Каких термов? Почему этих?);
    \item вывод по результатам очередного шага и дальнейшие действия.
\end{itemize}

Т.к. резольвента хранится в виде стека, то состояние резольвенты требуется отображать в столбик: вершина -- сверху! Новый шаг надо начинать с нового состояния резольвенты!

\clearpage

\listingfile{main.pro}{1-1}{Prolog}{Реализация базы знаний}{linerange={1-37}}

В Таблице \ref{tbl:1-1} представлен порядок поиска ответа на вопрос 2.

\begin{landscape}
    \setlength{\LTcapwidth}{\linewidth}
    \begin{longtable}{|c|c|c|c|c|}
        \caption[Порядок формирования результата для 1-го вопроса]{Порядок формирования результата для 1-го вопроса} \label{tbl:1-1}\\
    
        \hline
            Шаг & Сравниваемые термы; & Дальнейшие & Резольвента & Подстановка \\
                & результаты & действия & & \\
        \endfirsthead
    
        \multicolumn{5}{l}
        {{\tablename\ \thetable{} -- продолжение}} \\
        \hline 
            Шаг & Сравниваемые термы; & Дальнейшие & Резольвента & Подстановка \\
                & результаты & действия & & \\\hline
        \endhead
        
        \hline \multicolumn{5}{|r|}{{Продолжение на следующей странице}} \\ \hline
        \endfoot
        
        \hline \multicolumn{5}{|r|}{{Конец таблицы}} \\ \hline
        \endlastfoot
        
        \hline
              & grand("Andrey"{}, NameMaGrands, & Прямой ход & grand("Andrey"{}, & \\
            1 &  "w"{}, \_) и par("Andrey"{}, & Переход к & NameMaGrands, "w"{}, \_) &\\
              & "Boris"{}, "m"{}) & след. предл. & &\\
			  & Главные функторы не равны & & & \\
			\hline
			\dots & \dots & \dots & \dots & \dots \\
			\hline 
			  & grand("Andrey"{}, NameMaGrands, & Прямой ход & par("Andrey"{}, NameParent, "w"{}) & Child = "Andrey"{}\\
            7 &  "w"{}, \_) и grand(Child & & par(NameParent, NameGrand, Sex) & Line = "w"{} \\
              & NameGrand, Line, Sex) & &  &\\
			\hline
              & par("Andrey"{}, NameParent, "w"{}) & Прямой ход & par("Andrey"{}, NameParent, "w"{}) & Child = "Andrey"{}\\
            8 & par("Andrey"{}, "Boris"{}, "m"{}) & Переход к & par(NameParent, NameGrand, Sex) & Line = "w"{} \\
			  & Не унифицируемы & след. предл. & & \\
			\hline
              & par("Andrey"{}, NameParent, "w"{}) & Прямой ход & par("Daria"{}, & Child = "Andrey"{}\\
            9 & par("Andrey"{}, "Daria"{}, "w"{}) & & NameGrand, Sex), & Line = "w"{} \\
              &  & & & NameParent = "Daria"{}\\
            \hline
              & par("Daria"{}, NameGrand, Sex) & Прямой ход & par("Daria"{}, NameGrand, Sex) & Child = "Andrey"{}\\
            10 & par("Andrey"{}, "Boris"{}, "m"{}) & Переход к & & Line = "w"{} \\
              & Не унифицируемы & след. предл. & & NameParent = "Daria"{}\\
			\hline
			\dots & \dots & \dots & \dots & \dots \\
            \hline
              & par("Daria"{}, NameGrand, Sex) & Нашли ответ & par("Daria"{}, NameGrand, Sex) & Child = "Andrey"{}\\
            14 & par("Daria"{}, "Fedor"{}, "m"{}) & & & Line = "w"{} \\
              & & & & NameParent = "Daria"{}\\
              & & & & NameGrand = "Fedor"{}\\
            \hline
              & par("Daria"{}, NameGrand, Sex) & Нашли ответ & par("Daria"{}, NameGrand, Sex) & Child = "Andrey"{}\\
            15 & par("Daria"{}, "Julia"{}, "m"{}) & & & Line = "w"{} \\
              & & & & NameParent = "Daria"{}\\
              & & & & NameGrand = "Julia"{}\\
            \hline
			\dots & \dots & \dots & \dots & \dots \\
			\hline
              & par("Andrey"{}, NameParent, "w"{}) & Завершение & par("Andrey"{}, NameParent, "w"{}) & Child = "Andrey"{} \\
            21 &grand(Child, NameGrand, & работы & & Line = "w" \\
              & Line, Sex) & 2 подст. & & \\
              & Главные функторы не равны & в рез-те & & \\
    \end{longtable}
\end{landscape}
