\chapter{Контрольный вопросы}

\section{В каком случае система запускает алгоритм унификации?}

Система запускает алгоритм унификации автоматически при необходимости что-то доказать.

\section{Каковы назначение и результат использования алгоритма унификации?}

Унификация -- механизм логического вывода. Результат -- подстановка.

\section{Какое первое состояние резольвенты?}

Заданный вопрос (goal).

\section{Как меняется резольвента?}

Преобразования резольвенты выполняются с помощью редукции.
Редукцией цели G с помощью программы P называется замена цели G телом
того правила из P, заголовок которого унифицируется с целью.
Новая резольвента образуется в два этапа:
\begin{itemize}
    \item в текущей резольвенте выбирается одна из подцелей и для неё выполняется редукция;
    \item к полученной конъюнкции целей применяется подстановка, полученная как наибольший общий унификатор цели и заголовка сопоставленного с ней правила.
\end{itemize}

\section{В каких пределах программы уникальны переменные?}

Именованная переменная уникальна в рамках предложения, в котором она используется.

\section{Как применяется подстановка, полученная с помощью алгоритма унификации?}

Полученная с помощью алгоритма унификации подстановка применяется к целям в резольвенте.

\section{В каких случаях запускается механизм отката?}

Механизм отката запустится в случае неудачи алгоритма унификации.