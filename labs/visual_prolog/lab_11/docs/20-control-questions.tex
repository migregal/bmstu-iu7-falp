\chapter{Контрольный вопросы}

\section{Вопросы 11}
Программа на языке \texttt{Prolog} представляет собой базу знаний и вопрос. База знаний --- набор фактов и правил, которые формируют базу знаний о предметной области. Факт --- частный случай правила, состоит только из заголовка и с его помощью фиксируется \textit{истиностное} отношение между объектами предметной области. С помощью правила также фиксируются знания, однако правила обладают телом, в котором фиксируется условие истинности правила. При поиске ответа на вопрос \texttt{Prolog} рассматривает альтернативные варианты и находит все возможные решения --- множества значений переменных, при которых на поставленный вопрос можно ответить \texttt{''да''}.

Программа состоит из разделов (структура программы), каждый имеет свой заголовок:
\begin{itemize}
    \item \texttt{constants} --- раздел описания констант.
    \item \texttt{domains} --- раздел описания доменов.
    \item \texttt{database} --- раздел описания предикатов внутренней базы данных.
    \item \texttt{predicates} --- раздел описания предикатов.
    \item \texttt{clauses} --- раздел описания предложений базы знаний.
    \item \texttt{goal} --- раздел описания внутренней цели (вопроса).
\end{itemize}

В программе не обязательно должны быть описаны все разделы.

\section{Вопросы 11(2)}
\begin{enumerate}
    \item \texttt{student(Name, Surname, "MSU").}
        При сравнении вопроса с 2 предложениями базы знаний унификация вопроса и предложения базы знаний проходит успешно: совпадает функтор, арность, успешно унифицируются все аргументы (а переменные \texttt{Name} и \texttt{Surname} конкретизируются значениями аргументов функтора предложения БЗ, стоящими на тех же позициях соответственно, и возвращаются в качестве решений)
    \item \texttt{student(''Andrey'', ''Andreev'', University). }
        Аналогичная ситуация, только переменная \texttt{University} в этой ситуации конкретизируется значениями \textit{BMSTU} и \textit{MSU}
    \item \texttt{student(Name, Surname, "ITMO").}
        Унификация вопроса не проходит ни с одним предложением базы знаний (не проходит унификация третьего аргумента, потому что для 2 констант унификация успешно проходит только при их совпадении, а в базе знаний нет функтора \texttt{student} с 3 аргументом \texttt{''ITMO''})
\end{enumerate}