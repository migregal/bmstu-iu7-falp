\chapter{Контрольный вопросы}

\section{Элементы языка: определение, синтаксис, представление в памяти}

\subsection*{Определение элементов языка}

Элементы языка -- атомы и точечные пары (структуры, которые строятся с помощью унифицированных структур - блоков памяти - бинарных узлов). Атомы бывают:
\begin{itemize}
    \item \textbf{символы} (идентификаторы) -- синтаксически представляют собой набор литер (последовательность букв и цифр, начинающаяся с буквы; могут быть связанные и несвязанные);
    \item \textbf{специальные символы} -- используются для обозначения <<логических>> констант (\texttt{T}, \texttt{Nil});
    \item \textbf{самоопределимые атомы} -- числа, строки - последовательность символов в кавычках (\texttt{"abc"}).
\end{itemize}

\subsection*{Синтаксис элементов языка}

\texttt{S-выражение ::= <атом> | <точечная пара>}

\texttt{Точечная пара ::= (<атом> . <атом>) \\
\indent\indent| (<точечная пара> . <атом>) \\
\indent\indent| (<атом> . <точечная пара>) \\
\indent\indent| (<точечная пара> . <точечная пара>)}

\texttt{Список ::= <пустой список> | <непустой список>}, где 

\texttt{<пустой список> ::= () | Nil},

\texttt{<непустой список> ::= (<S-выражение>. <список>)},

Список -- частный случай S-выражения.

Синтаксически любая структура (точечная пара или список) заключается в круглые скобки:
\begin{itemize}
    \item \texttt{(A . B)} -- точечная пара;
    \item \texttt{(A)} -- список из одного элемента;
    \item Непустой список -- \texttt{(A . (B . (C . (D . Nil))))} или \texttt{(A B C D)};
    \item Пустой список -- \texttt{Nil} или \texttt{()}.
\end{itemize}

Элементы списка могут быть списками, например -- \texttt{(A (B C) (D (E)))}. Таким образом, синтаксически наличие скобок является признаком структуры -- списка или точечной пары.

\subsection*{Представление элементов языка в памяти}

Любая непустая структура Lisp в памяти представляется списковой ячейкой, хранящий два указателя: на голову (первый элемент) и хвост (все остальное).

\section{Особенности языка Lisp. Структура программы. Символ апостроф}

\subsection*{Особенности языка \texttt{Lisp}}

К особенностям языка \texttt{Lisp} относятся:
\begin{itemize}
    \item список является базовой конструкцией языка;
    \item отсутствие типов данных (так как все представляется S-выражением);
    \item такая трактовка логических значений, при которой истинным считается любое выражение (атом, список), отличное от \texttt{Nil};
    \item автоматическое динамическое распределение памяти;
    \item единая синтаксическая форма записи программ и данных, что позволяет обрабатывать структуры данных как программы и модифицировать программы как данные;
    \item все операции над данными оформляются и записываются как функции, которые имеют значение, даже если их основное предназначение -- осуществление некоторого побочного эффекта (например, определение новой функции);
\end{itemize}

\subsection*{Структура программы}

Программы в \texttt{Lisp} понимают как применение функции к ее аргументам (вызов функции). Аргументом функции может быть любая форма \texttt{Lisp}. Список, первый элемент
которого -- представление функции, остальные элементы -- аргументы функции, -- это основная конструкция в \texttt{Lisp}-программе.

Формат: (функция аргумент1 аргумент2 ... )

Названия (имена) функций изображаются с помощью атомов (для наглядности можно предпочитать заглавные буквы: \\\texttt{CONS}, \texttt{CAR}, \texttt{CDR}, \texttt{ATOM}, \texttt{EQ}, \dots

Все более сложные формы в \texttt{Lisp} понимают как применение функции к ее аргументам (вызов функции). Аргументом функции может быть любая форма.

\subsection*{Символ апостроф}

Символ апостроф воспринимается как спецальная функция \texttt{quote}. Данная функция блокирует вычисления своего единственного аргумента, то есть он воспринимается как константа. При выполнении функции аргумент обрабатывается по общей схеме.

\section{Базис языка. Ядро языка}

\subsection*{Базис языка}

Базис состоит из:
\begin{enumerate}
    \item структуры, атомы;
    \item встроенные (примитивные) функции (\texttt{atom}, \texttt{eq}, \texttt{cons}, \texttt{car}, \texttt{cdr});
    \item специальные функции и функционалы, управляющие обработкой структур, представляющих вычислимые выражения (\texttt{quote}, \texttt{cond}, \texttt{lambda}, \texttt{label}, \texttt{eval}).
\end{enumerate}

\subsection*{Ядро языка}

Ядро языка было создано в 60-х годах прошлого века известным ученым Дж. Маккарти для решения задач обработки символьной информации.

Помимо Базиса языка, ядро включает в себя наиболее часто используемые функции, не входящие в Базис.