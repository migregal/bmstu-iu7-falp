\chapter{Контрольный вопросы}

\section{Cтруктуроразрушающие и не разрушающие структуру списка функции}

\subsection{Не разрушающие структуру списка функции}

Данные функции не меняют сам объект-аргумент, а создают копию.

\subsubsection{Функция \texttt{append}}

Объединяет списки. Это форма, можно передать больше 2 аргументов. Создает копию для всех аргументов, кроме последнего.

Пример: \texttt{(append '(1 2) '(3 4))} -- \texttt{(1 2 3 4)}.

\subsubsection{Функция \texttt{reverse}}

Возвращает копию исходного списка, элементы которого переставлены в обратном порядке. \textbf{В целях эффективности работает только на верхнем уровне}.

Пример: \texttt{(reverse '(1 2 3 4))} -- \texttt{(4 3 2 1)}.

\subsubsection{Функция \texttt{remove}}

Модифицирует, но работает с копией, поэтому не разрушает. Данная функция удаляет элемент по значению (Часто разрушающая аналогичная функция называется \texttt{delete}). По умолчанию используется \texttt{eql} для сравнения на равенство, но можно передать другую функцию через ключевой параметр \texttt{:test}.

Примеры: 
\begin{enumerate}
    \item \texttt{(remove 3 '(1 2 3))} -- \texttt{(1 2)};
    \item \texttt{(remove '(1 2) '((1 2) (3 4)))} -- \texttt{((1 2) (3 4))};
    \item \texttt{(remove '(1 2) '((1 2) (3 4)) :test \#'equal)} -- \texttt{((3 4))};
\end{enumerate}

\subsubsection{Функция \texttt{rplaca}}

Переставляет \texttt{car}-указатель на 2 элемент-аргумент (\textit{S}-выражение).

Пример: \texttt{(rplaca '(1 2 3) 3)} -- \texttt{(3 2 3)}.

\subsubsection{Функция \texttt{rplacd}}

Переставляет \texttt{cdr}-указатель на 2 элемент-аргумент (\textit{S}-выражение).

Пример: \texttt{(rplacd '(1 2 3) '(4 5))} -- \texttt{(1 4 5)}.

\subsubsection{Функция \texttt{subst}}

Заменяет все элементы списка, которые равны 2-ому переданному элементу-аргументу на 1-ый элемент-аргумент. \textit{По умолчанию для сравнения используется функция \texttt{eql}}.

Пример: \texttt{(subst 2 1 '(1 2 1 3))} -- \texttt{(2 2 2 3)}.

\subsection{Структуроразрушающие функции}

Данные функции меняют сам объект-аргумент, невозможно вернуться к исходному списку. Чаще всего такие функции начинаются с префикса \texttt{n-}.

\subsubsection{Функция \texttt{nconc}}

Работает аналогично \texttt{append}, только не копирует свои аргументы, а разрушает структуру.

\subsubsection{Функция \texttt{nreverse}}

Работает аналогично \texttt{reverse}, но не создает копии.

\subsubsection{Функция \texttt{nsubst}}

Работает аналогично функции \texttt{nsubst}, но не создает копии.

\section{Отличие в работе функций cons, list, append, nconc и в их результате}

Функция \texttt{cons} -- чисто математическая, конструирует списковую ячейку, которая может вовсе и не быть списком (будет списком только в том случае, если 2 аргументом передан список).

Примеры:
\begin{enumerate}
    \item \texttt{(cons 2 '(1 2))} -- \texttt{(2 1 2)} -- список;
    \item \texttt{(cons 2 3)} -- \texttt{(2 . 3)} -- не список.
\end{enumerate}

Функция \texttt{list} -- форма, принимает произвольное количество аргументов и конструирует из них список. Результат -- всегда список. При нуле аргументов возвращает пустой список.

Примеры:
\begin{enumerate}
    \item \texttt{(list 1 2 3)} -- \texttt{(1 2 3)};
    \item \texttt{(list 2 '(1 2))} -- \texttt{(2 (1 2))};
    \item \texttt{(list '(1 2) '(3 4))} -- \texttt{((1 2) (3 4))};
\end{enumerate}

Функция \texttt{append} -- форма, принимает на вход произвольное количество аргументов и для всех аргументов, кроме последнего, создает копию, ссылая при этом последний элемент каждого списка-аргумента на первый элемент следующего по порядку списка-аргумента (так как модифицируются все списки-аргументы, кроме последнего, копирование для последнего не делается в целях эффективности).

Примеры:
\begin{enumerate}
    \item \texttt{(append '(1 2) '(3 4))} -- \texttt{(1 2 3 4)};
    \item \texttt{(append '((1 2) (3 4)) '(5 6))} -- \texttt{((1 2) (3 4) 5 6)}.
\end{enumerate}

Функция \texttt{nconc} работает аналогично \texttt{append}, но не копирует свои аргументы, а разрушает структуру.