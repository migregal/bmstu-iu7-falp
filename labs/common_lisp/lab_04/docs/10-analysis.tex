\chapter{Практическая часть}

\section{Чем принципиально отличаются функции cons, list, append?}

Пусть:
\begin{lstlisting} [
	float=h!,
	frame=single,
	numbers=left,
	abovecaptionskip=-5pt,
	caption={объявление функций из условия},
	label={lst:1-1},
	language={Lisp},
]
(setf lst1 '(a b))
(setf lst2 '(c d))
\end{lstlisting}

В Таблице \ref{tbl:1-1} приведены результаты вычисления выражений.

\begin{table}[!ht]
	\begin{center}
		\caption{Результаты вычисления выражений}
		\label{tbl:1-1}
		\begin{tabular}{|l|l|}
			\hline
			\bfseries Выражение & \bfseries Результат\\\hline
            \texttt{(cons lst1 lst2)} & \texttt{((A B) C D)}\\\hline
            \texttt{(list lst1 lst2)} & \texttt{((A B) (C D))}\\\hline
            \texttt{(append lst1 lst2)} & \texttt{(A B C D)}\\\hline
		\end{tabular}
	\end{center}
\end{table}

\section{Каковы результаты вычисления следующих выражений?}

В Таблице \ref{tbl:2-1} приведены результаты вычисления выражений.

\begin{table}[!ht]
	\begin{center}
		\caption{Результаты вычисления выражений}
		\label{tbl:2-1}
		\begin{tabular}{|l|l|}
			\hline
			\bfseries Выражение & \bfseries Результат\\\hline
            \texttt{(reverse ())} & \texttt{(Nil)}\\\hline
            \texttt{(last ())} & \texttt{(Nil)}\\\hline
            \texttt{(reverse '(a))} & \texttt{(a)}\\\hline
            \texttt{(last '(a))} & \texttt{(a)}\\\hline
            \texttt{(reverse '((a b c)))} & \texttt{((a b c))}\\\hline
            \texttt{(last '((a b c)))} & \texttt{((a b c))}\\\hline
		\end{tabular}
	\end{center}
\end{table}

\section{Написать, по крайней мере, два варианта функции, которая возвращает последний элемент своего списка-аргумента}

\subsection{Рекурсия 1}

\begin{lstlisting} [
	float=h!,
	frame=single,
	numbers=left,
	abovecaptionskip=-5pt,
	caption={объявление функций из условия},
	label={lst:3-1},
	language={Lisp},
]
(defun my-last-recursive-internal (lst)
  (if (cdr lst) 
   (my-last-recursive-internal (cdr lst))
   (car lst)))
(defun my-last-recursive (lst)
  (and lst (my-last-recursive-internal lst)))
\end{lstlisting}

\subsection{Рекурсия 2}

\begin{lstlisting} [
	float=h!,
	frame=single,
	numbers=left,
	abovecaptionskip=-5pt,
	caption={объявление функций из условия},
	label={lst:3-2},
	language={Lisp},
]
(defun my-last-recursive-internal-2 (lst last-el)
  (if (eql nil lst) 
    last-el 
    (my-last-recursive-internal-2 (cdr lst) (car lst))))
(defun my-last-recursive-2 (lst)
  (my-last-recursive-internal-2 lst nil))
\end{lstlisting}

\subsection{С помощью \texttt{reduce}}

\begin{lstlisting} [
	float=h!,
	frame=single,
	numbers=left,
	abovecaptionskip=-5pt,
	caption={объявление функций из условия},
	label={lst:3-3},
	language={Lisp},
]
(defun my-last-reduce (lst)
  (reduce #'(lambda (acc el) el) lst))
\end{lstlisting}

\clearpage

\section{Написать, по крайней мере, два варианта функции, которая возвращает свой список-аргумент без последнего элемента}

\subsection{Рекурсия}

\begin{lstlisting} [
	float=h!,
	frame=single,
	numbers=left,
	abovecaptionskip=-5pt,
	caption={объявление функций из условия},
	label={lst:4-1},
	language={Lisp},
]
(defun no-last-internal (lst acc)
  (if (cdr lst)
   (no-last-internal (cdr lst) (cons (car lst) acc))
   (nreverse acc)))
(defun no-last (lst)
  (and lst (no-last-internal lst nil)))
\end{lstlisting}

\subsection{Функции ядра}

\begin{lstlisting} [
	float=h!,
	frame=single,
	numbers=left,
	abovecaptionskip=-5pt,
	caption={объявление функций из условия},
	label={lst:4-2},
	language={Lisp},
]
(defun no-last-kern (lst)
  (and lst (nreverse (cdr (reverse lst)))))
\end{lstlisting}

\section{Написать простой вариант игры в кости, в котором бросаются две правильные кости}

Если сумма выпавших очков равна 7 или 11 -- выигрыш, если выпало (1,1) или (6,6) --- игрок имеет право снова бросить кости, во всех остальных случаях ход переходит ко второму игроку, но запоминается сумма выпавших очков. Если второй игрок не выигрывает абсолютно, то выигрывает тот игрок, у которого больше очков. Результат игры и значения выпавших костей выводить на экран с помощью функции print.

\listingfile{dices.lisp}{lst:dices}{Lisp}{Игра в кости}{}
