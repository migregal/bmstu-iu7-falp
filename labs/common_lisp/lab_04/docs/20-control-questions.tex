\chapter{Контрольный вопросы}

\section{Базис языка}

Базис состоит из:
\begin{enumerate}
    \item структуры, атомы;
    \item встроенные (примитивные) функции (\texttt{atom}, \texttt{eq}, \texttt{cons}, \texttt{car}, \texttt{cdr});
    \item специальные функции и функционалы, управляющие обработкой структур, представляющих вычислимые выражения (\texttt{quote}, \texttt{cond}, \texttt{lambda}, \texttt{label}, \texttt{eval}).
\end{enumerate}

\section{Классификация функций}

Функции в \texttt{Lisp} классифицируют следующим образом:

\begin{itemize}
    \item чистые математические функции;
    \item рекурсивные функции;
    \item специальные функции --- формы (сегодня 2 аргумента, завтра - 5);
    \item псевдофункции (создают эффект на внешнем устройстве);
    \item функции с вариативными значениями, из которых выбирается 1;
    \item функции высших порядков --- функционал: используется для синтаксического управления программ (абстракция языка).
\end{itemize}

По назначению функции разделяются следующим образом:

\begin{itemize}
    \item конструкторы --- создают значение (\texttt{cons}, например);
    \item селекторы --- получают доступ по адресу (\texttt{car}, \texttt{cdr});
    \item предикаты --- возвращают \texttt{Nil}, \texttt{T}.
    \item функции сравнения -- такие как: \texttt{eq}, \texttt{eql}, \texttt{equal}, \texttt{equalp}.
\end{itemize}

\section{Способы создания функций}

Функции в \texttt{Lisp} можно задавать следующими способами:

\subsection*{\texttt{Lambda}-выражение}

Синтаксис:

\texttt{(lambda <$\lambda$-список> форма)}

Пример:

\begin{lstlisting} [
	float=h!,
	frame=single,
	numbers=left,
	abovecaptionskip=-5pt,
	caption={Функция определенная Lambda-выражением},
	label={cc:1},
	language={Lisp},
]
(lambda (a b) (sqrt (+ (* a a) (* b b))))
\end{lstlisting}

\subsection*{Именованая функция}

Синтаксис:

\texttt{(defun <имя функции> <$\lambda$-выражение>)}

Пример:

\begin{lstlisting} [
	float=h!,
	frame=single,
	numbers=left,
	abovecaptionskip=-5pt,
	caption={Функция определенная Lambda-выражением},
	label={cc:2},
	language={Lisp},
]
(defun hyp (a b) (sqrt (+ (* a a) (* b b))))
\end{lstlisting}

\section{Работа функций \texttt{and}, \texttt{or}, \texttt{if}, \texttt{cond}}

\subsection{Функция \texttt{and}}

Синтаксис:
\begin{lstlisting} [
	float=h!,
	frame=single,
	numbers=left,
	abovecaptionskip=-5pt,
	caption={функция \texttt{and}},
	label={сс:3-1},
	language={Lisp},
]
(and expression-1 expression-2 ... expression-n)
\end{lstlisting}

Функция возвращает первое expression, результат вычисления которого = \texttt{Nil}. Если все не \texttt{Nil}, то возвращается результат вычисления последнего выражения.

\clearpage

Примеры:

\begin{lstlisting} [
	float=h!,
	frame=single,
	numbers=left,
	abovecaptionskip=-5pt,
	caption={пример использования \texttt{and}},
	label={с:3-2},
	language={Lisp},
]
(and 1 Nil 2)
\end{lstlisting}

Результат: \texttt{Nil}

\begin{lstlisting} [
	float=h!,
	frame=single,
	numbers=left,
	abovecaptionskip=-5pt,
	caption={пример использования \texttt{and}},
	label={с:3-3},
	language={Lisp},
]
(and 1 2 3)
\end{lstlisting}

Результат: 3

\subsection{Функция \texttt{or}}

Синтаксис:
\begin{lstlisting} [
	float=h!,
	frame=single,
	numbers=left,
	abovecaptionskip=-5pt,
	caption={функция \texttt{or}},
	label={с:3-4},
	language={Lisp},
]
(or expression-1 expression-2 ... expression-n)
\end{lstlisting}

Функция возвращает первое expression, результат вычисления которого не \texttt{Nil}. Если все \texttt{Nil}, то возвращается \texttt{Nil}.

Примеры:

\begin{lstlisting} [
	float=h!,
	frame=single,
	numbers=left,
	abovecaptionskip=-5pt,
	caption={пример использования \texttt{or}},
	label={с:3-5},
	language={Lisp},
]
(or Nil Nil 2)
\end{lstlisting}

Результат: 2

\begin{lstlisting} [
	float=h!,
	frame=single,
	numbers=left,
	abovecaptionskip=-5pt,
	caption={пример использования \texttt{or}},
	label={с:3-6},
	language={Lisp},
]
(or 1 2 3)
\end{lstlisting}

Результат: 1

\subsection{Функция \texttt{if}}

Синтаксис:

\begin{lstlisting} [
	float=h!,
	frame=single,
	numbers=left,
	abovecaptionskip=-5pt,
	caption={функция \texttt{if}},
	label={с:3-7},
	language={Lisp},
]
(if condition t-expression f-expression)
\end{lstlisting}

Если вычисленный предикат не \texttt{Nil}, то выполняется \texttt{t-expression}, иначе - \texttt{f-expression}.

Примеры:

\begin{lstlisting} [
	float=h!,
	frame=single,
	numbers=left,
	abovecaptionskip=-5pt,
	caption={пример использования \texttt{if}},
	label={с:3-8},
	language={Lisp},
]
(if Nil 2 3)
\end{lstlisting}

Результат: 3

\begin{lstlisting} [
	float=h!,
	frame=single,
	numbers=left,
	abovecaptionskip=-5pt,
	caption={пример использования \texttt{if}},
	label={с:3-9},
	language={Lisp},
]
(if 0 2 3)
\end{lstlisting}

Результат: 2

\subsection{Функция \texttt{cond}}

Синтаксис:

\begin{lstlisting} [
	float=h!,
	frame=single,
	numbers=left,
	abovecaptionskip=-5pt,
	caption={Функция \texttt{cond}},
	label={с:3-10},
	language={Lisp},
]
(cond 
  (condition-1 expression-1)
  (condition-2 expression-2)
  ...
  (condition-n expression-n))
\end{lstlisting}

По порядку вычисляются и проверяются на равенство с \texttt{Nil} предикаты. Для первого предиката, который не равен \texttt{Nil}, вычисляется находящееся с ним в списке выражение и возвращается его значение. Если все предкаты вернут \texttt{Nil}, то и \texttt{cond} вернет \texttt{Nil}.

Примеры:

\begin{lstlisting} [
	float=h!,
	frame=single,
	numbers=left,
	abovecaptionskip=-5pt,
	caption={Пример использования \texttt{cond}},
	label={с:3-11},
	language={Lisp},
]
(cond (Nil 1) (2 3))
\end{lstlisting}

Результат: 3

\begin{lstlisting} [
	float=h!,
	frame=single,
	numbers=left,
	abovecaptionskip=-5pt,
	caption={Пример использования \texttt{cond}},
	label={с:3-12},
	language={Lisp},
]
(cond (Nil 1) (Nil 2))
\end{lstlisting}

Результат: \texttt{Nil}
