\chapter{Практическая часть}

\section{Написать хвостовую рекурсивную функцию my-reverse, которая развернет верхний уровень своего списка-аргумента lst}

\begin{lstlisting} [
	float=h!,
	frame=single,
	numbers=left,
	abovecaptionskip=-5pt,
	caption={Функция, разворачивающая первый уровень списка},
	label={lst:1-1},
	language={Lisp},
]
(defun my-reverse-internal (lst res)
  (cond ((null lst) res)
        ((my-reverse (cdr lst) (cons (car lst) res)))) )

(defun my-reverse (lst res)
  (my-reverse-internal lst ()))
\end{lstlisting}

\section{Написать функцию, которая возвращает первый элемент списка -аргумента, который сам является непустым списком}

\begin{lstlisting} [
	float=h!,
	frame=single,
	numbers=left,
	abovecaptionskip=-5pt,
	caption={Функция, возвращающая первый элемент списка, являющийся списком},
	label={lst:2-1},
	language={Lisp},
]
(defun get-first-sublist (lst)
  (cond
    ((null lst) nil)
    ((cond ((listp (car lst)) (car lst))))
    ((get-first-sublist (cdr lst)))))
\end{lstlisting}

\clearpage

\section{Написать функцию, которая выбирает из заданного списка только те числа, которые больше 1 и меньше 10. (Вариант: между двумя заданными границами)}

\begin{lstlisting} [
	float=h!,
	frame=single,
	numbers=left,
	abovecaptionskip=-5pt,
	caption={Функция, выбирающая из списка только те числа между двумя заданными границами},
	label={lst:3-1},
	language={Lisp},
]
(defun all-between-internal (lst a b result)
  (cond
    ((null lst) result)
    ((all-between-internal
        (cdr lst)
        a
        b
        (cond
          ((cond
            ((>= (car lst) a) (<= (car lst) b)))
             (cons (car lst) result))
          (result))))))

(defun all-between (lst a b)
  (all-between-internal lst a b nil))
\end{lstlisting}

\section{Напишите рекурсивную функцию, которая умножает на заданное число-аргумент все числа из заданного списка-аргумента, когда...}

\subsection{все элементы списка -- числа}

\begin{lstlisting} [
	float=h!,
	frame=single,
	numbers=left,
	abovecaptionskip=-5pt,
	caption={Функция, умножающая на заданное число все числа из списка чисел},
	label={lst:4-1},
	language={Lisp},
]
(defun mulall-internal (lst num res)
  (cond ((null lst) res)
        ((mulall-internal
            (cdr lst)
            num
            (cons (* num (car lst)) res)))))

(defun mulall (lst num)
  (mulall-internal lst num ()))
\end{lstlisting}

\clearpage

\subsection{элементы списка -- любые объекты}

\begin{lstlisting} [
	float=h!,
	frame=single,
	numbers=left,
	abovecaptionskip=-5pt,
	caption={Функция, умножающая на заданное число все числа из списка},
	label={lst:4-2},
	language={Lisp},
]
(defun mulall-internal (lst num res)
  (cond
    ((null lst) res)
    ((symbolp (car lst))
      (mulall-internal (cdr lst) num (cons (car lst) res)))
    ((numberp (car lst))
      (mulall-internal
        (cdr lst)
        num
        (cons (* num (car lst)) res)))
    ((consp (car lst))
      (mulall-internal
        (cdr lst)
        num
        (cons
          (mulall-internal (car lst) num ())
          res)))))

(defun mulall (lst num)
  (mulall-internal lst num ()))
\end{lstlisting}

\section{Напишите функцию, select-between, которая из списка-аргумента, содержащего только числа, выбирает только те, которые расположены между двумя указанными границами аргументами и возвращает их в виде списка}

\begin{lstlisting} [
	float=h!,
	frame=single,
	numbers=left,
	abovecaptionskip=-5pt,
	caption={Функция, вычисляющая сумму чисел заданного одноуровневого смешанного списка},
	label={lst:5-1},
	language={Lisp},
]
(defun check-border (x a b)
  (and (>= x a) (<= x b)) )

(defun select-between (lst a b)
  (cond
    ((null lst) ())
    ((check-border (car lst) a b)
      (cons (car lst) (select-between (cdr lst) a b)))
    ((select-between (cdr lst) a b))))
\end{lstlisting}

\section{Написать рекурсивную версию (с именем rec-add) вычисления суммы чисел заданного списка: ...}

\subsection{одноуровневого смешанного}

\begin{lstlisting} [
	float=h!,
	frame=single,
	numbers=left,
	abovecaptionskip=-5pt,
	caption={Функция, вычисляющая сумму чисел заданного одноуровневого смешанного списка},
	label={lst:6-1},
	language={Lisp},
]
(defun rec-add-internal (lst sum)
  (cond
    ((null lst) sum)
    ((rec-add-internal
      (cdr lst)
      (cond
        ((numberp (car lst)) (+ sum (car lst)))
        (sum))))))

(defun rec-add (lst)
	(rec-add-internal lst 0))
\end{lstlisting}

\subsection{структурированного}

\begin{lstlisting} [
	float=h!,
	frame=single,
	numbers=left,
	abovecaptionskip=-5pt,
	caption={Функция, вычисляющая сумму чисел заданного структурированного списка},
	label={lst:6-2},
	language={Lisp},
]
(defun rec-add-internal (lst sum)
  (cond
    ((null lst) sum)
    ((numberp (car lst))
      (rec-add-internal (cdr lst) (+ sum (car lst))))
    ((null (car lst))
      (rec-add-internal (cdr lst) sum))
    ((rec-add-internal
      (cons (caar lst) (cons (cdar lst) (cdr lst)))
      sum))))

(defun rec-add (lst)
  (rec-add-internal lst 0))
\end{lstlisting}

\clearpage

\section{Написать рекурсивную версию с именем recnth функции nth}

\begin{lstlisting} [
	float=h!,
	frame=single,
	numbers=left,
	abovecaptionskip=-5pt,
	caption={Рекурсивная реализация функции nth},
	label={lst:7-1},
	language={Lisp},
]
(defun rec-nth (n lst)
  (cond
    ((null lst) nil)
    ((= n 0) (car lst))
    ((rec-nth (- n 1) (cdr lst)))))
\end{lstlisting}

\section{Написать рекурсивную функцию allodd, которая возвращает t когда все элементы списка нечетные}

\begin{lstlisting} [
	float=h!,
	frame=single,
	numbers=left,
	abovecaptionskip=-5pt,
	caption={Функция, проверяющая, являются ли все элементы списка нечетными},
	label={lst:8-1},
	language={Lisp},
]
(defun alloddr (lst)
  (cond
    ((null lst) T)
    ((oddp (car lst)) (alloddr (cdr lst)))))        
\end{lstlisting}

\section{Написать рекурсивную функцию, которая возвращает первое нечетное число из списка (структурированного), возможно создавая некоторые вспомогательные функции}


\begin{lstlisting} [
	float=h!,
	frame=single,
	numbers=left,
	abovecaptionskip=-5pt,
	caption={Функция, возвращающая первый нечетный элемент структурированного списка},
	label={lst:9-1},
	language={Lisp},
]
(defun first-odd (lst)
  (cond
    ((null lst) nil)
    ((numberp (car lst))
      (cond 
        ((oddp (car lst)) (car lst))
        ((first-odd (cdr lst)))))
    ((consp (car lst))
      (first-odd (cons (caar lst) (cons (cdar lst) (cdr lst)))))
    ((first-odd (cdr lst)))))
\end{lstlisting}

\section{Используя cons-дополняемую рекурсию с одним тестом завершения, написать функцию которая получает как аргумент список чисел, а возвращает список квадратов этих чисел в том же порядке}

\begin{lstlisting} [
	float=h!,
	frame=single,
	numbers=left,
	abovecaptionskip=-5pt,
	caption={Функция, возводящая все элементы списка чисел в квадрат},
	label={lst:10-1},
	language={Lisp},
]
(defun square-lst (lst)
  (cond
    ((null lst) Nil)
    ((cons
      (* (car lst) (car lst))
      (square-lst (cdr lst))))))
\end{lstlisting}
