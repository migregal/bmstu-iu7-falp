\chapter{Практическая часть}

\section{Представить следующие списки в виде списочных ячеек (№1)}

\subsection{\texttt{`(open close halph)}}

\imgw{1-1}{h!}{0.5\textwidth}{Списочные ячейки для \texttt{`(open close halph)}}

\subsection{\texttt{`((open1) (close2) (halph3))}}

\imgw{1-2}{h!}{0.6\textwidth}{Списочные ячейки для \texttt{`((open1) (close2) (halph3))}}

\subsection{\texttt{`((one) for all (and (me (for you))))}}

\imgw{1-3}{h!}{\textwidth}{Списочные ячейки для \texttt{`((one) for all (and (me (for you))))}}

\subsection{\texttt{`((TOOL) (call))}}

\imgw{1-4}{h!}{0.4\textwidth}{Списочные ячейки для \texttt{`((TOOL) (call))}}

\subsection{\texttt{`((TOOL1) ((call)) ((sell)))}}

\imgw{1-5}{h!}{0.6\textwidth}{Списочные ячейки для}

\subsection{\texttt{`(((TOOL) (call)) ((sell)))}}

\imgw{1-6}{h!}{0.6\textwidth}{Списочные ячейки для}

\clearpage

\section{Используя только функции \texttt{CAR} и \texttt{CDR}, написать выражения, возвращающие: (№2)}

\subsection{Второй элемент заданного списка}

\begin{lstlisting} [
		float=h!,
		frame=single,
		numbers=left,
		abovecaptionskip=-5pt,
		caption={Выражение, возвращающее второй элемент заданного списка},
		language={Lisp},
	]
(car (cdr `(A B C D)))
\end{lstlisting}

\subsection{Третий элемент заданного списка}

\begin{lstlisting} [
		float=h!,
		frame=single,
		numbers=left,
		abovecaptionskip=-5pt,
		caption={Выражение, возвращающее третий элемент заданного списка},
		language={Lisp},
	]
(car (cdr (cdr `(A B C D))))
\end{lstlisting}

\subsection{Четвертый элемент заданного списка}

\begin{lstlisting} [
		float=h!,
		frame=single,
		numbers=left,
		abovecaptionskip=-5pt,
		caption={Выражение, возвращающее четвертый элемент заданного списка},
		language={Lisp},
	]
(car (cdr (cdr (cdr `(A B C D)))))
\end{lstlisting}

\section{Что будет в результате вычисления выражений? (№3)}


\begin{lstlisting} [
		float=h!,
		frame=single,
		numbers=left,
		abovecaptionskip=-5pt,
		caption={Выражение 1},
		language={Lisp},
	]
(caadr `((blue cube) (red pyramid)))
\end{lstlisting}

Результат: \texttt{red}.

\begin{lstlisting} [
		float=h!,
		frame=single,
		numbers=left,
		abovecaptionskip=-5pt,
		caption={Выражение 2},
		language={Lisp},
	]
(cdar `((abc) (def) (ghi)))
\end{lstlisting}

Результат: \texttt{Nil}

\begin{lstlisting} [
		float=h!,
		frame=single,
		numbers=left,
		abovecaptionskip=-5pt,
		caption={Выражение 3},
		language={Lisp},
	]
(cadr `((abc) (def) (ghi)))
\end{lstlisting}

Результат: \texttt{(def)}
\clearpage

\begin{lstlisting} [
		float=h!,
		frame=single,
		numbers=left,
		abovecaptionskip=-5pt,
		caption={Выражение 4},
		language={Lisp},
	]
(caddr `((abc) (def) (ghi)))
\end{lstlisting}

Результат: \texttt{(ghi)}

\section{Что будет в результате вычисления выражении? (№4)}

\begin{table}[ht]
	\begin{center}
		\caption{Результаты вычисления выражений}
		\begin{tabular}{|l|l|}
			\hline
			\bfseries Выражение & \bfseries Результат \\\hline
            \texttt{(list `Fred `and `Wilma)} & \texttt{(Fred and Wilma)} \\\hline
            \texttt{(cons `Fred `(and Wilma))} & \texttt{(Fred and Wilma)} \\\hline
            \texttt{(list `Fred `(and Wilma))} & \texttt{(Fred (and Wilma))} \\\hline
            \texttt{(cons `Fred `(Wilma))} & \texttt{(Fred Wilma)} \\\hline
            \texttt{(cons Nil Nil)} & \texttt{(Nil)}\\\hline
            \texttt{(list Nil Nil)} & \texttt{(Nil Nil)}\\\hline
            \texttt{(cons T Nil)} & \texttt{(T)} \\\hline
            \texttt{(list T Nil)} & \texttt{(T Nil)} \\\hline
            \texttt{(cons Nil T)} & \texttt{(Nil . T)}\\\hline
            \texttt{(list Nil T)} & \texttt{(Nil T)}\\\hline
            \texttt{(list Nil)} & \texttt{(Nil)}\\\hline
            \texttt{(cons T (list Nil))} & \texttt{(T Nil)} \\\hline
            \texttt{(cons `(T) Nil)} & \texttt{((T))} \\\hline
            \texttt{(list `(T) Nil)} & \texttt{((T) Nil)} \\\hline
            \texttt{(list `(one two) `(free temp))} & \texttt{((one two) (free temp))} \\\hline
            \texttt{(cons `(one two) `(free temp))} & \texttt{((one two) free temp)} \\\hline
		\end{tabular}
	\end{center}
\end{table}

\clearpage
    
\section{Написать lambda-выражение и соответствующую функцию возвращающие список\dots (№5)}

\begin{itemize}
    \item Функция \texttt{(f ar1 ar2 ar3 ar4)}, возвращающая список следующего\\ вида:\\\texttt{((ar1 ar2) (ar3 ar4))};
    
    \begin{lstlisting} [
		float=h!,
		frame=single,
		numbers=left,
		abovecaptionskip=-5pt,
		caption={lambda-выражение},
		label={lst:2-1},
		language={Lisp},
	]
    (lambda (ar1 ar2 ar3 ar4) 
        (list (list ar1 ar2) (list ar3 ar4))
    )
    \end{lstlisting}
    
    \begin{lstlisting} [
		float=h!,
		frame=single,
		numbers=left,
		abovecaptionskip=-5pt,
		caption={функция, использующая \texttt{list}},
		label={lst:2-2},
		language={Lisp},
	]
    (defun f (ar1 ar2 ar3 ar4) 
        (list (list ar1 ar2) (list ar3 ar4))
    )
    \end{lstlisting}

    \begin{lstlisting} [
		float=h!,
		frame=single,
		numbers=left,
		abovecaptionskip=-5pt,
		caption={функция, использующая \texttt{cons}},
		label={lst:2-3},
		language={Lisp},
	]
    (defun f (ar1 ar2 ar3 ar4) 
        (cons 
            (cons ar1 (cons ar2 Nil)) 
            (cons 
                (cons ar3 (cons ar4 Nil)) 
                Nil
            )
        )
    )
    \end{lstlisting}

    \item Функция \texttt{(f ar1 ar2)}, возвращающая список следующего вида:\\\texttt{((ar1) (ar3))};
    
    \begin{lstlisting} [
		float=h!,
		frame=single,
		numbers=left,
		abovecaptionskip=-5pt,
		caption={lambda-выражение},
		label={lst:2-4},
		language={Lisp},
	]
    (lambda (ar1 ar2) (list (list ar1) (list ar2)))
    \end{lstlisting}
    
    \begin{lstlisting} [
		float=h!,
		frame=single,
		numbers=left,
		abovecaptionskip=-5pt,
		caption={функция, использующая \texttt{list}},
		label={lst:2-5},
		language={Lisp},
	]
    (defun f (ar1 ar2) (list (list ar1) (list ar2)))
    \end{lstlisting}
    
    \clearpage
    
    \begin{lstlisting} [
		float=h!,
		frame=single,
		numbers=left,
		abovecaptionskip=-5pt,
		caption={функция, использующая \texttt{cons}},
		label={lst:2-6},
		language={Lisp},
	]
    (defun f (ar1 ar2) 
        (cons
            (cons ar1 Nil)
            (cons (cons ar2 Nil) Nil)
        )
    )
    \end{lstlisting}
    
    \item Функция \texttt{(f ar1)}, возвращающая список следующего вида:\\\texttt{(((ar1)))};
    
    \begin{lstlisting} [
		float=h!,
		frame=single,
		numbers=left,
		abovecaptionskip=-5pt,
		caption={lambda-выражение},
		label={lst:2-7},
		language={Lisp},
	]
    (lambda (ar1) (list (list (list ar1))))
    \end{lstlisting}
    
    \begin{lstlisting} [
		float=h!,
		frame=single,
		numbers=left,
		abovecaptionskip=-5pt,
		caption={функция, использующая \texttt{list}},
		label={lst:2-8},
		language={Lisp},
	]
    (defun f (ar1) (list (list (list ar1))))
    \end{lstlisting}
    
    \begin{lstlisting} [
		float=h!,
		frame=single,
		numbers=left,
		abovecaptionskip=-5pt,
		caption={функция, использующая \texttt{cons}},
		label={lst:2-9},
		language={Lisp},
	]
    (defun f (ar1) (cons (cons (cons ar1 Nil) Nil) Nil))
    \end{lstlisting}
    
    \item Представить результаты в виде списочных ячеек:
    
    \imgw{2-1}{h!}{0.9\textwidth}{Списочные ячейки для листинга \ref{lst:2-1}}
    
    \imgw{2-2}{h!}{\textwidth}{Списочные ячейки для листинга \ref{lst:2-2}}
    
    \imgw{2-3}{h!}{\textwidth}{Списочные ячейки для листинга \ref{lst:2-3}}
    
    \imgw{2-4}{h!}{\textwidth}{Списочные ячейки для листинга \ref{lst:2-4}}
    
    \imgw{2-5}{h!}{\textwidth}{Списочные ячейки для листинга \ref{lst:2-5}}
    
    \imgw{2-6}{h!}{\textwidth}{Списочные ячейки для листинга \ref{lst:2-6}}
    
    \imgw{2-7}{h!}{\textwidth}{Списочные ячейки для листинга \ref{lst:2-7}}
    
    \imgw{2-8}{h!}{\textwidth}{Списочные ячейки для листинга \ref{lst:2-8}}

    \imgw{2-9}{h!}{\textwidth}{Списочные ячейки для листинга \ref{lst:2-9}}    
    
\end{itemize}