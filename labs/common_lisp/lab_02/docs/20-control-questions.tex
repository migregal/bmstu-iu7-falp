\chapter{Контрольный вопросы}

\section{Базис языка}

Базис состоит из:
\begin{enumerate}
    \item структуры, атомы;
    \item встроенные (примитивные) функции (\texttt{atom}, \texttt{eq}, \texttt{cons}, \texttt{car}, \texttt{cdr});
    \item специальные функции и функционалы, управляющие обработкой структур, представляющих вычислимые выражения (\texttt{quote}, \texttt{cond}, \texttt{lambda}, \texttt{label}, \texttt{eval}).
\end{enumerate}

\section{Классификация функций}

Функции в \texttt{Lisp} классифицируют следующим образом:

\begin{itemize}
    \item чистые математические функции;
    \item рекурсивные функции;
    \item специальные функции --- формы (сегодня 2 аргумента, завтра - 5);
    \item псевдофункции (создают эффект на внешнем устройстве);
    \item функции с вариативными значениями, из которых выбирается 1;
    \item функции высших порядков --- функционал: используется для синтаксического управления программ (абстракция языка).
\end{itemize}

По назначению функции разделяются следующим образом:

\begin{itemize}
    \item конструкторы --- создают значение (\texttt{cons}, например);
    \item селекторы --- получают доступ по адресу (\texttt{car}, \texttt{cdr});
    \item предикаты --- возвращают \texttt{Nil}, \texttt{T}.
    \item функции сравнения -- такие как: \texttt{eq}, \texttt{eql}, \texttt{equal}, \texttt{equalp}.
\end{itemize}

\section{Способы создания функций}

Функции в \texttt{Lisp} можно задавать следующими способами:
 
\subsection*{\texttt{Lambda}-выражение}

Синтаксис:

\texttt{(lambda <$\lambda$-список> форма)}

Пример:

\begin{lstlisting} [
	float=h!,
	frame=single,
	numbers=left,
	abovecaptionskip=-5pt,
	caption={Функция определенная Lambda-выражением},
	label={cc:1},
	language={Lisp},
]
(lambda (a b) (sqrt (+ (* a a) (* b b))))
\end{lstlisting}

\subsection*{Именованная функция}

Синтаксис:

\texttt{(defun <имя функции> <$\lambda$-выражение>)}

Пример:

\begin{lstlisting} [
	float=h!,
	frame=single,
	numbers=left,
	abovecaptionskip=-5pt,
	caption={Пример определения именованной функции},
	label={cc:2},
	language={Lisp},
]
(defun hyp (a b) (sqrt (+ (* a a) (* b b))))
\end{lstlisting}

\section{Функции \texttt{car} и \texttt{cdr}}

\texttt{car} --- функция получения первого элемента точечной пары.

Примеры:

\begin{table}[!ht]
    \small
	\begin{center}
		\begin{tabular}{|c|c|c|}
            \hline
            \bfseries S-выражение & \bfseries Результат выполнения \texttt{car} \\\hline
            (A . B) & A \\\hline
            ((A . B) . C) & (A . B) \\\hline
            A & ошибка \\\hline
		\end{tabular}
	\end{center}
\end{table}

\texttt{cdr} --- функция получения второго элемента точечной пары.

\begin{table}[!ht]
    \small
	\begin{center}
		\begin{tabular}{|c|c|c|}
            \hline
            \bfseries S-выражение & \bfseries Результат выполнения \texttt{cdr} \\\hline
            (A . B) & B \\ \hline
            (A . (B . C)) & (B . C) \\\hline
            A & ошибка \\\hline
		\end{tabular}
	\end{center}
\end{table}

\section{Назначение и отличие \texttt{list} от \texttt{cons}}

\texttt{cons} --- функция конструирования точечной пары, на вход получает 2 значения и делает из них точечную пару.

\texttt{list} --- функция конструирования списка. На вход получает произвольное количество элементов и делает из них список.

Вызовы \texttt{(list 1 2 3 4)} и \texttt{(cons 1 (cons 2 (cons 3 (cons 4 Nil))))} эквивалентны, то есть дают одинаковый результат.