\chapter{Лекция 2}

\section{Представление атома}

Атом представляется 5-тью указателями:
\begin{itemize}
    \item name;
    \item value;
    \item function;
    \item properties;
    \item package.
\end{itemize}

\section{Классификация функций}

Любые объекты можно классифицировать множеством способов, предпочитая ту или иную характеристику.

Чистые функции -- это чистые математические функции (так, как принято в математике).

Аргументами и результататми функций является S-выражение.

Способы классификации функций:
\begin{itemize}
    \item рекурсивные -- в связи с отстутствием переменных, повторные вычисления, в основном, реализуются при помощи рекурсии;
    \item специальные функции (формы) -- от вызова к вызову меняется число аргументов. Кроме того, данные функции обрабатываются по-разному;
    \item псевдофункции -- создают спецэффекты (например -- вывод на экран);
    \item функции с вариантами значений;
    \item функционалы -- в качестве аргумента используют другие функции или возвращают функцию в качестве результата.
\end{itemize}

Второй вариант:
\begin{itemize}
    \item базисные -- минимальный набор функций, которые позволяют решить любую задачу (\texttt{car}, \texttt{cdr}, \texttt{cons}, \texttt{atom}, а так же формы и функционалы);
\end{itemize}

Базовое определение функции -- это lambda-выражение.

Регулярная программа на \texttt{Lisp} -- функция. Первый элемент -- представление функции.

В базис включены формы: apply, funcall.

\section{Способы определения функций}

Базисный -- lambda--выражение.

$\lambda$-выражение:

\texttt{(lambda <$\lambda$-список> форма)}

Вызов:

\texttt{(<$\lambda$-выражение> <последовательность форм>)}

Именованная функция:

\texttt{(defun f <$\lambda$-выражение>)}

\texttt{(defun f (x_1 \dots x_n) (форма))}

\texttt{(let (x_1 p_1) (x_2 p_2) \dots (x_n p_n)) e === ((lambda (x1 x2 \dots x_n) e) p1 p2 \dots p_n)}

Классификация базисных функций и функций ядра:
\begin{itemize}
    \item селкторы -- \texttt{car} и \texttt{cdr};
    \item конструкторы -- \texttt{cons} (базисная) и \texttt{list} (форма);
    \item предикаты -- логические функции: \texttt{atom}, \texttt{null}, \tex\texttt{eq}ttt{listp}, \texttt{consp};
    \item функции сравнения -- \texttt{eq}, \texttt{eql}, \texttt{equal}, \texttt{equalp}.
\end{itemize}

\section{Выполнение программ}

Программа -- это вызов функции на внешнем уровне. Программа является S-выражением.
