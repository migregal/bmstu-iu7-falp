\chapter{Лекция 1}

\section{Список литературы}

\begin{itemize}
    \item Основы функционального программирования -- Л. В. Городняя;
    \item \texttt{ANSI Common Lisp} -- Пол Грэм, м. 2014.
\end{itemize}

\section{Вводные определения}

Базис языка -- минимальный набор конструкций языка и структур данных, позволяющий решить любую задачу.

При загрузке системы загружается ядро -- не только базисные, но и наиболее часто употребляемые функции (встроенные функции).

Функциональные языки опираются на лямбда исчисления Черча, поэтому понятие функции -- базовое.

Принцип работы компьютера -- пошаговое изменения состояния вычислителя.

Наличие оператора присваивания -- признак императивных зыков программирования.

\section{Особенности Lisp}

\texttt{Lisp} -- List processor -- создан в 1960-е.

\texttt{Lisp} -- безтиповый язык.

В \texttt{Lisp}:
\begin{itemize}
    \item чисто реализованы идеи функционального программирования;
    \item предложена символьная обработка, поэтому нет разделения между текстом программы и данными (т.е. \texttt{Lisp} -- интерпретатор);
    \item все можно представить в виде композиции функций;
    \item базовая конструкция - список, поэтому работает исключительно на указателях.
\end{itemize}

\section{Элементы языка}

Элементы языка -- атомы и точечные пары (структуры, которые строятся с помощью унифицированных структур - блоков памяти - бинарных узлов). Атомы бывают:
\begin{itemize}
    \item \textbf{символы} (идентификаторы) -- синтаксически представляют собой набор литер (последовательность букв и цифр, начинающаяся с буквы; могут быть связанные и несвязанные);
    \item \textbf{специальные символы} -- используются для обозначения <<логических>> констант (\texttt{T}, \texttt{Nil});
    \item \textbf{самоопределимые атомы} -- числа, строки - последовательность символов в кавычках (\texttt{"abc"}).
\end{itemize}


\section{Синтаксис элементов языка}

\texttt{Точечная пара ::= (<атом> . <атом>) | (<точечная пара> . <атом>) | (<атом> . <точечная пара>) | (<точечная пара> . <точечная пара>)}

\texttt{Список ::= <пустой список> | <непустой список>}, где

\texttt{<пустой список> ::= () | Nil},

\texttt{<непустой список> ::= (<S-выражение>. <список>)},

S-выражение -- это атом илизаключенная в скобки пара из двух S-выражений, разделенных точкой.

Список -- частный случай S-выражения.

Список представляется в виде 2-х указателей.

Синтаксически любая структура (точечная пара или список) заключается в круглые скобки:
\texttt{(A . B)} -- точечная пара.
\texttt{(A)} -- список из одного элемента.
Непустой список -- \texttt{(A . (B . (C . (D . Nil))))} или \texttt{(A B C D)}
Пустой список -- \texttt{Nil} или \texttt{()}.

Элементы списка могут быть списками, например -- \texttt((A (B C) (D (E)))). Таким образом, синтаксически наличие скобок является признаком структуры -- списка или точечной пары.

Любая непустая структура Lisp в памяти представляется  списковой ячейкой, хранящий два указателя: на голову (первый элемент) и хвост (все остальное).
